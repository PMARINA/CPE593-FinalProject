%%%%%%%%%%%%%%%%%%%%%%%%%%%%%%%%%%%%%%%%%
% Homework Assignment Article
% LaTeX Template
% Version 1.3.5r (2018-02-16)
%
% This template has been downloaded from:
% /cl.uni-heidelberg.de/~zimmermann/
%
% Original author:
% Victor Zimmermann (zimmermann@cl.uni-heidelberg.de)
%
% License:
% CC BY-SA 4.0 (https://creativecommons.org/licenses/by-sa/4.0/)
%
%%%%%%%%%%%%%%%%%%%%%%%%%%%%%%%%%%%%%%%%%

%----------------------------------------------------------------------------------------

\documentclass[letterpaper,12pt]{article} % Uses article class in A4 format

%----------------------------------------------------------------------------------------
%   FORMATTING
%----------------------------------------------------------------------------------------

\setlength{\parskip}{0pt}
\setlength{\parindent}{0pt}
\setlength{\voffset}{-15pt}

%----------------------------------------------------------------------------------------
%   PACKAGES AND OTHER DOCUMENT CONFIGURATIONS
%----------------------------------------------------------------------------------------

\usepackage[letterpaper, margin=1in]{geometry} % Sets margin to 2.5cm for A4 Paper
\usepackage[onehalfspacing]{setspace} % Sets Spacing to 1.5
\usepackage{tikz}
\usepackage{pgfplots}

% \usepackage[T1]{fontenc} % Use European encoding
% \usepackage[utf8]{inputenc} % Use UTF-8 encoding
% \usepackage{inconsolata} % Use the Charter font


\usepackage{microtype} % Slightly tweak font spacing for aesthetics

\usepackage[english]{babel} % Language hyphenation and typographical rules

\usepackage{amsthm, amsmath, amssymb} % Mathematical typesetting
\usepackage{marvosym, wasysym} % More symbols
\usepackage{float} % Improved interface for floating objects
\usepackage[final, colorlinks = true, 
            linkcolor = black, 
            citecolor = black,
            urlcolor = black]{hyperref} % For hyperlinks in the PDF
\usepackage{graphicx, multicol} % Enhanced support for graphics
\usepackage{xcolor} % Driver-independent color extensions
\usepackage{rotating} % Rotation tools
\usepackage{listings, style/lstlisting} % Environment for non-formatted code, !uses style file!
\usepackage{pseudocode} % Environment for specifying algorithms in a natural way
\usepackage{style/avm} % Environment for f-structures, !uses style file!
\usepackage{booktabs} % Enhances quality of tables
% \usepackage{xcolor}

\usepackage{tikz-qtree} % Easy tree drawing tool
\tikzset{every tree node/.style={align=center,anchor=north},
         level distance=2cm} % Configuration for q-trees
\usepackage{style/btree} % Configuration for b-trees and b+-trees, !uses style file!

\usepackage{titlesec} % Allows customization of titles
% \renewcommand\thesection{\arabic{section}.} % Arabic numerals for the sections
% \titleformat{\section}{\large}{\thesection}{1em}{}
% \renewcommand\thesubsection{\alph{subsection})} % Alphabetic numerals for subsections
% \titleformat{\subsection}{\large}{\thesubsection}{1em}{}
% \renewcommand\thesubsubsection{\roman{subsubsection}.} % Roman numbering for subsubsections
\titleformat{\subsubsection}{\large}{\thesubsubsection}{1em}{}

\usepackage[all]{nowidow} % Removes widows

\usepackage[backend=biber,style=numeric,
            sorting=nyt, natbib=true]{biblatex} % Complete reimplementation of bibliographic facilities
\addbibresource{main.bib}
\usepackage{csquotes} % Context sensitive quotation facilities

\usepackage[yyyymmdd]{datetime} % Uses YEAR-MONTH-DAY format for dates
\renewcommand{\dateseparator}{-} % Sets dateseparator to '-'

\usepackage{fancyhdr} % Headers and footers
\pagestyle{fancy} % All pages have headers and footers
\fancyhead{}\renewcommand{\headrulewidth}{0pt} % Blank out the default header
% \fancyfoot[L]{\textsc{I pledge my honor that I have abided by the Stevens Honor System.}} % Custom footer text
\fancyfoot[C]{} % Custom footer text
\fancyfoot[R]{\thepage} % Custom footer text
\usepackage[color=green]{todonotes}
\newcommand{\note}[1]{\marginpar{\scriptsize \textcolor{red}{#1}}} % Enables comments in red on margin
\newcommand {\E}[1]{\cdot 10^{#1}}
\usepackage{enumitem}
% \usepackage{inconsolata}
% \renewcommand*\familydefault{\ttdefault} %% Only if the base font of the document is to be typewriter style
% \usepackage[T1]{fontenc}
% \usepackage{mathpazo}
% \renewcommand\rmdefault{hpv}% or whatever
% \usepackage{fontspec}
% \usepackage{unicode-math}
% \setmathfont{consolas}
% \setmainfont{Libertinus Serif}[Numbers=SlashedZero]
% \setmathfont{Libertinus Math}

% Steal the slashed 0 from the text font
% \setmathfont{Libertunis Serif}[range={"0030},Numbers={SlashedZero}]
%----------------------------------------------------------------------------------------

\begin{document}
% \newfontfamily{\ttconsolas}{Consolas}
% \setmonofont{Consolas}%\usepackage{fontspec}

%----------------------------------------------------------------------------------------
%   TITLE SECTION
%----------------------------------------------------------------------------------------

\title{template_assignment} % Article title
\fancyhead[C]{}
\begin{minipage}{0.295\textwidth} % Left side of title section
    \raggedright
    CPE 593\\ % Your lecture or course
    \footnotesize % Authors text size
    %\hfill\\ % Uncomment if right minipage has more lines
    Pridhvi Myneni, 10435884 % Your name, your matriculation number
    \medskip\hrule
\end{minipage}
\begin{minipage}{0.4\textwidth} % Center of title section
    \centering
    \large % Title text size
    RK45 Notes\\ % Assignment title and number
    \normalsize % Subtitle text size
    Fall 2020\\ % Assignment subtitle
\end{minipage}
\begin{minipage}{0.295\textwidth} % Right side of title section
    \raggedleft
    \today\\ % Date
    \footnotesize % Email text size
    %\hfill\\ % Uncomment if left minipage has more lines
    pmyneni@stevens.edu% Your email
    \medskip\hrule
\end{minipage}

%----------------------------------------------------------------------------------------
%   ARTICLE CONTENTS
%----------------------------------------------------------------------------------------

\tableofcontents
\section{About RK45}
\newcommand{\target}[0]{{\int_{x_0}}^{x_1}(y(x))dx}
\renewcommand{\target}[0]{y(x_1)}
\newcommand{\currentEquation}[1]{(#1-3)^5 + (#1-3)^4 + (#1-3)^3 + (#1-3)^2}
\newcommand{\currEq}[0]{\currentEquation{x}}
RK represents the Runge-Kutta (RK) class of integrators for estimating the value of an equation.
For the purposes of this article, we will be assuming that we are attempting to find \(y(x_1)\) from \(y(x_0)\).
% \[\target\]
Additionally, although this equation, \(y(x)\), is only a function of one variable, these methods can be applied to multi-dimensional equations.
\subsection{First Order RK}
The first order of the RK class is simply the Euler approximation to the equation. 
\[\target \approx y(x_0) + \Delta_x * y'(x_0) \text{, where } \Delta_x = x_1 - x_0\]
\begin{figure}[H]
  \centering
  \begin{tikzpicture}
    \begin{axis}[
      domain=1.8:3.4,
      samples=500,
      smooth,
      no markers,
      ]
      \addplot[blue] {\currEq};
      \addplot +[mark=none, black] coordinates {(2.5, -0.8) (2.5, 0.3)};
      \addplot +[mark=none, black] coordinates {(3, -0.8) (3, 0.3)};
      \addplot[red]{-0.4375*x + 1.25};
    \end{axis}
  \end{tikzpicture}
\caption{\(\currEq\) with first derivative shown, and the approximation at \(x_1 = 3\), given \(x_0 = 2.5\).}
\end{figure}

Although the inaccuracy in this can be fairly large unless a sufficiently small timestep (\(\Delta_x\)) is used, there is a large benefit in terms of computational power required to compute this result. 
To calculate this normally, one would have to compute first, \(x-3\) and store that in a register. 
Next, compute \((x-3)^5\), during which the other powers will be computed.
1 computation for each of the following operations produces 8 operations to evaluate the given equation at a value.
\begin{enumerate}
  \item \(a = x-3\)
  \item \(b = a*a\)
  \item \(c = a*b\)
  \item \(d = a*c\)
  \item \(e = a*d\)
  \item \(f = b + c\)
  \item \(f = f + d\)
  \item \(y(x_1) = f + e\)
\end{enumerate}

Meanwhile, using the first order RK, and a first order taylor series, we are able to approximate the value of \(y(x_1)\) based on \(y(x_0)\) by doing the following.
\begin{enumerate}
  \item Given \(y(x_0)\)... 
  \item \todo[inline]{How do we calculate the derivative?}
  \item \todo[inline]{Is this only more efficient not for higher order polynomials, but rather more complex equations, such as when sin/cos or fractions with polynomials everywhere exist?}
  \item Somehow calculate the derivative
  \item \(\Delta_y = \Delta_x * \frac{dy}{dx}\)
  \item \(y(x_1) = y(x_0) + \Delta_y\)
\end{enumerate}
% here be dragons
%---------------------------------
% fin
\end{document}
%---------------------------------